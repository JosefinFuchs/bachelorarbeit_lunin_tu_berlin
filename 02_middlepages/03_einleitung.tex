\chapter{Einleitung}
\section{Abstrakt}
Es werden Röntgenabsorptionsspektren in Transmission aufgenommen, die neben den Absorptionskanten auch eine sichtbare Feinstruktur enthalten. Diese entsteht durch die Anwesenheit von benachbarten Atomen. Die Besonderheit des demonstrierten Messverfahrens liegt darin, dass eine Röntgenröhre als Quelle der Röntgenstrahlung, statt einer monochromatischen Synchrotron Quelle, dient. Das lässt die Messungen in kürzester Zeit unter Laborbedingungen mit einem Spektrometer in der von Hamos Geometrie durchführen. Die Intensität ist trotzdem ausreichend, da ein hochgeglühter pyrolytischer Graphit (HAPG,  highly annealed pyrolitic Graphite) Mosaikkristall verwendet wird, wodurch die Bragg-Bedingung an vielen Stellen im Kristall zur gleichen Zeit erfüllt ist. Es wird die Bismut L$_\mathrm{III}$ Kante bei $E=\SI{13418}{\eV}$ und die Nickel K-Kante bei $E=\SI{8332}{\eV}$ von verschiedenen chemischen Verbindungen untersucht. Durch die unterschiedlichen Bindungen kann eine Verschiebung der Kante beobachtet werden. Durch Linearkombinationen der Feinstruktur von verschiedenen Referenzproben werden mögliche Bindungsverhältnisse von unbekannten Proben bestimmt. Außerdem wird die Realisierbarkeit der Aufnahme von Absorptionsspektren von Proben in Quarzkapillaren, die neue In-Situ Untersuchungsmethoden ermöglichen würden, untersucht.
Es werden Röntgenabsorptionsspektren in Transmission aufgenommen, die neben den Absorptionskanten auch eine sichtbare Feinstruktur enthalten. Diese entsteht durch die Anwesenheit von benachbarten Atomen. Die Besonderheit des demonstrierten Messverfahrens liegt darin, dass eine Röntgenröhre als Quelle der Röntgenstrahlung, statt einer monochromatischen Synchrotron Quelle, dient. Das lässt die Messungen in kürzester Zeit unter Laborbedingungen mit einem Spektrometer in der von Hamos Geometrie durchführen. Die Intensität ist trotzdem ausreichend, da ein hochgeglühter pyrolytischer Graphit (HAPG,  highly annealed pyrolitic Graphite) Mosaikkristall verwendet wird, wodurch die Bragg-Bedingung an vielen Stellen im Kristall zur gleichen Zeit erfüllt ist. Es wird die Bismut L$_\mathrm{III}$ Kante bei $E=\SI{13418}{\eV}$ und die Nickel K-Kante bei $E=\SI{8332}{\eV}$ von verschiedenen chemischen Verbindungen untersucht. Durch die unterschiedlichen Bindungen kann eine Verschiebung der Kante beobachtet werden. Durch Linearkombinationen der Feinstruktur von verschiedenen Referenzproben werden mögliche Bindungsverhältnisse von unbekannten Proben bestimmt. Außerdem wird die Realisierbarkeit der Aufnahme von Absorptionsspektren von Proben in Quarzkapillaren, die neue In-Situ Untersuchungsmethoden ermöglichen würden, untersucht.
Es werden Röntgenabsorptionsspektren in Transmission aufgenommen, die neben den Absorptionskanten auch eine sichtbare Feinstruktur enthalten. Diese entsteht durch die Anwesenheit von benachbarten Atomen. Die Besonderheit des demonstrierten Messverfahrens liegt darin, dass eine Röntgenröhre als Quelle der Röntgenstrahlung, statt einer monochromatischen Synchrotron Quelle, dient. Das lässt die Messungen in kürzester Zeit unter Laborbedingungen mit einem Spektrometer in der von Hamos Geometrie durchführen. Die Intensität ist trotzdem ausreichend, da ein hochgeglühter pyrolytischer Graphit (HAPG,  highly annealed pyrolitic Graphite) Mosaikkristall verwendet wird, wodurch die Bragg-Bedingung an vielen Stellen im Kristall zur gleichen Zeit erfüllt ist. Es wird die Bismut L$_\mathrm{III}$ Kante bei $E=\SI{13418}{\eV}$ und die Nickel K-Kante bei $E=\SI{8332}{\eV}$ von verschiedenen chemischen Verbindungen untersucht. Durch die unterschiedlichen Bindungen kann eine Verschiebung der Kante beobachtet werden. Durch Linearkombinationen der Feinstruktur von verschiedenen Referenzproben werden mögliche Bindungsverhältnisse von unbekannten Proben bestimmt. Außerdem wird die Realisierbarkeit der Aufnahme von Absorptionsspektren von Proben in Quarzkapillaren, die neue In-Situ Untersuchungsmethoden ermöglichen würden, untersucht.
Es werden Röntgenabsorptionsspektren in Transmission aufgenommen, die neben den Absorptionskanten auch eine sichtbare Feinstruktur enthalten. Diese entsteht durch die Anwesenheit von benachbarten Atomen. Die Besonderheit des demonstrierten Messverfahrens liegt darin, dass eine Röntgenröhre als Quelle der Röntgenstrahlung, statt einer monochromatischen Synchrotron Quelle, dient. Das lässt die Messungen in kürzester Zeit unter Laborbedingungen mit einem Spektrometer in der von Hamos Geometrie durchführen. Die Intensität ist trotzdem ausreichend, da ein hochgeglühter pyrolytischer Graphit (HAPG,  highly annealed pyrolitic Graphite) Mosaikkristall verwendet wird, wodurch die Bragg-Bedingung an vielen Stellen im Kristall zur gleichen Zeit erfüllt ist. Es wird die Bismut L$_\mathrm{III}$ Kante bei $E=\SI{13418}{\eV}$ und die Nickel K-Kante bei $E=\SI{8332}{\eV}$ von verschiedenen chemischen Verbindungen untersucht. Durch die unterschiedlichen Bindungen kann eine Verschiebung der Kante beobachtet werden. Durch Linearkombinationen der Feinstruktur von verschiedenen Referenzproben werden mögliche Bindungsverhältnisse von unbekannten Proben bestimmt. Außerdem wird die Realisierbarkeit der Aufnahme von Absorptionsspektren von Proben in Quarzkapillaren, die neue In-Situ Untersuchungsmethoden ermöglichen würden, untersucht.
Es werden Röntgenabsorptionsspektren in Transmission aufgenommen, die neben den Absorptionskanten auch eine sichtbare Feinstruktur enthalten. Diese entsteht durch die Anwesenheit von benachbarten Atomen. Die Besonderheit des demonstrierten Messverfahrens liegt darin, dass eine Röntgenröhre als Quelle der Röntgenstrahlung, statt einer monochromatischen Synchrotron Quelle, dient. Das lässt die Messungen in kürzester Zeit unter Laborbedingungen mit einem Spektrometer in der von Hamos Geometrie durchführen. Die Intensität ist trotzdem ausreichend, da ein hochgeglühter pyrolytischer Graphit (HAPG,  highly annealed pyrolitic Graphite) Mosaikkristall verwendet wird, wodurch die Bragg-Bedingung an vielen Stellen im Kristall zur gleichen Zeit erfüllt ist. Es wird die Bismut L$_\mathrm{III}$ Kante bei $E=\SI{13418}{\eV}$ und die Nickel K-Kante bei $E=\SI{8332}{\eV}$ von verschiedenen chemischen Verbindungen untersucht. Durch die unterschiedlichen Bindungen kann eine Verschiebung der Kante beobachtet werden. Durch Linearkombinationen der Feinstruktur von verschiedenen Referenzproben werden mögliche Bindungsverhältnisse von unbekannten Proben bestimmt. Außerdem wird die Realisierbarkeit der Aufnahme von Absorptionsspektren von Proben in Quarzkapillaren, die neue In-Situ Untersuchungsmethoden ermöglichen würden, untersucht.
\section{Test von section}
Das Buch \cite{cartier_micron_2014} ist toll!
\newpage
\section{neuer section}
Der Test läuft weiter \cite{tripathi_dichroic_2011}