\newacronym[user1=\emph{engl. „plasma x-ray source“}]{pxs}{PXS}{Laser getriebene Plasma-Röntgenquelle}
\newacronym[user1=\emph{engl. „refection zone plate“}]{rzp}{RZP}{Reflexionszonenplatten}
\chapter{Laser-getriebenes Instrument für resonante Streuung mit weichen Röntgenstrahlen}
\label{text:quelle_roentgen}
Als die Quelle der weichen Röntgenstrahlung in den weiteren Experimenten dient die \gls{pxs} \cite{schick_laser-driven_2021}. Deren Funktionsprinzip liegt die Emission von Wolfram im weichen Röntgenbereich zugrunde, die durch hochenergetische Laserpulsen getrieben wird \cite{mantouvalou_high_2015}. Das Spektrum von \gls{pxs} wurde mit einem kaliebrierten Spektrometer vermessen und ist in der Abbildung \ref{fig:xps_spectrum} zu sehen.

Der Strahl kann mithilfe der \gls{rzp} fokussiert werden. Zur Verfügung stehen nämlich die zwei \gls{rzp}, die für die Frequenzen Fe und Gd konstruiert wurden. 

Die zu untersuchende mehrschichtige Probe (s. Abschnitt \ref{text:streuung_theorie}), wie bereits erwähnt wurde, besteht aus Gd und Fe. Mithilfe der \gls{rzp} wird die Energie der M5 Gd-Absorptionskante $E_{Gd, M5} = \SI{1185(1)}{\eV}$ \cite[Abb. 6(a)]{prieto_x-ray_2005} auf die Probe abgebildet.

Justage mit ccd ascan und maximum of signal1/signal2 bezogen auf rzp\_phi 

% MOTIVATION SINGLE PHOTON COUNTING

% \subsection{Algorythmus 1}
% \subsection{Algorythmus 2}
% \subsection{Algorythmus 3}