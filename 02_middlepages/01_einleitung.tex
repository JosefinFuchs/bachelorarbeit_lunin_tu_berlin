\chapter{Einleitung}
\newacronym{fel}{FEL}{Freie-Elektronen-Laser}
\label{text:einleitung}
% Magnetische Effekte sind seit langem der Menschheit bekannt und spielten eine große Rolle bei der Entwicklung der menschlichen Zivilisation. So fanden diese Effekte ihre Anwendungen bereits im 11. Jahrhundert in Schiffsnavigation und machten bisher unvorstellbare Fernreisen möglich.

% \noindent
% Eine sprunghafte/großartige Entwicklung im Magnetismus Forschungsfeld fand im 19. Jahrhundert mit den Werken von Maxwell, Lorentz und Lenz statt. Somit wurde die Verwandtschaft der elektrischen und magnetischen Felder geklärt, die sich zunächst im 20. Jahrhundert auf quantenmechanischer Ebene in den Werken von Dirac, Bohr und Pauli aufgreifen ließ.

% \noindent
% Die digitale Revolution, allein die originale Erfindung eines Festplattenlaufwerkes, sei ohne Vorwissen zum Thema Magnetismus durchaus nicht möglich. Die Information dargestellt als binäre Folge wird in Form der alternierenden lokalen Magnetisierungen (\emph{magnetischen Domänen}) auf den drehenden Scheiben gespeichert. Der Wendepunkt bei den Datenspeicherungslösungen für Verbraucher und Unternehmen, und zwar die drastische Erhöhung der Datendichte, kam mit der Entdeckung des Effektes von Riesenmagnetowiderstand in den Dünnschichtsystemen. Für diese Entdeckung wurde man im Jahr 2007 mit dem Nobelpreis für Physik ausgezeichnet.
% \noindent
Die Entwicklung und Verständnis von Theorie des Magnetismus im 20. Jahrhundert fuhr zur theoretischen Begründung und Entdeckung der Magnetisierungsverteilung in Festkörper am mikroskopischen Niveau. So kann es zum Beispiel alternierende lokale Magnetisierungen (\emph{magnetischen Domänen}) entstehen. Zur Bildung der magnetischen Domänen und deren Charakteristiken tragen zahlreiche Faktoren wie Zusammensetzung und innere Struktur des Materials sowie Probetemperatur und äußeres angelegtes Magnetfeld bei.

\noindent
Es werden statische und dynamische Eigenschaften der Domänenbildung in Festkörpern und insbesondere in Dünnschichtsystemen in Bezug auf die Variation der äußeren Bedingungen heutzutage erforscht. Solche Prozesse können innerhalb von einigen Pikosekunden lang laufen. Daher sind Messmethoden mit einer zeitlichen Auflösung im Bereich von einigen Femtosekunden \cite{pfau_ultrafast_2012} gefragt, um die Dynamik des Prozesses zu untersuchen.

\noindent
Unter anderem betragen die charakteristischen Domänengrößen je nach Dünnschichtsystem \SIrange[range-units = single]{50}{300}{\nano\meter}. So ist die räumliche Auflösung in Hohe von einigen Nanometern gefragt. Es gibt mehrere Messverfahren, die den Zugang zur Forschung der magnetischen Effekten auf solchen Zeit- und Raumskalen ermöglichen. Ich möchte allerdings auf diejenigen fokussieren, deren die magneto-optischen Wechselwirkungen zugrunde liegen und zwar auf die Beugungsmethoden. Solche Untersuchungsmethoden sind sehr praktisch, da sie den direkten physischen Kontakt mit der Probe vermeiden und dadurch eine bessere Anpassung der Experimentalbedingungen ermöglichen.

\noindent
Zum Durchführen der Streuexperimenten an den mikroskopischen magnetischen Strukturen wird die weiche Röntgenstrahlung verwendet, \textcolor{red}{WEIL} \dots (Übergänge der zu untersuchenden Materialen in Dünnschichtsystemen in dem Bereich liegen, Beugung Wellenlänge und räumliche Auflösung).  Die vorausgesetzten Anforderungen an die Kohärenz und Brillanz der Quelle können hauptsächlich nur mithilfe von speziellen Anlagen wie Synchrotronstrahlungsquellen oder \gls{fel} erfüllt werden.  % kann ebenso für solche Messmethoden eingesetzt \cite{pfau_ultrafast_2012}

\noindent
Im Gegensatz zu der guten räumlichen Auflösung, welche die Synchrotronstrahlungsquellen anbieten, ist die zeitliche Auflösung für die Picosekunden-Prozesse nicht gut genug. So beträgt die Röntgenpulsdauer, also die Zeitauflösung, am Synchrotron PETRA III \SI{44}{\pico\second} (RMS), also \SI{100}{\pico\second} (FWHM), und wird mit der Frequenz ca. \SIrange[range-units = single]{5}{125}{\mega\hertz} \cite{noauthor_machine_nodate} emittiert.

\noindent
Die wesentlich kürzeren Pulsdauern sind mit einem \gls{fel} erreichbar. Als Beispiel einer der modernsten Anlagen dieser Art wird „XFEL“ betrachtet. Die Pulsdauern können in unterschiedlichen Betriebsmodi von \SI{25}{\femto\second} bis zu \SI{10}{\femto\second} (FWHM) eingestellt \cite{tschentscher_photon_2017}. Die Pulsfrequenz beträgt \SI{4,5}{\mega\hertz}. So eine hohe Frequenz ist nicht für manche Prozesse erforderlich und kann zur unerwünschten Veränderung durch zu hohe Energiezufuhr oder eben zur Zerstörung der zu untersuchende Probe führen. 

\noindent
Ein großer Nachteil liegt an den stark limitierten Messzeiten an \gls{fel}s, was die Anpassung der experimentellen Umgebung erschwert und die Möglichkeit der häufigen regelmäßigen Messungen ausschließt.

\noindent
Eine bevorzugte Alternative wäre, diese Art von Experimenten im Labor durchzuführen. Im Rahmen dieser Arbeit wird eine Röntgenquelle benutzt, in der Röntgenstrahlung aus dem mit einem Femtosekundenlaser angeregten Plasma erzeugt wird. Sie bietet eine mit \gls{fel} vergleichbare Zeitauflösung mit der Pulsdauer in Hohe von \SI{10}{\pico\second} (FWHM) und viel niedrigere Pulsfrequenz von \SI{100}{\hertz} \cite{schick_laser-driven_2021}, wodurch die dynamischen Messungen auf dem längeren Zeitintervall ohne Probezerstörung möglich sind. Nichtsdestotrotz hat so eine Röntgenquelle deutlich kleineren Photonenfluss, was viele technische Herausforderungen zum Messverfahren mit sich bringt.

%weiche röntgenstrahlung, kleiner Flux -> kleiner Kontrast
\noindent
Das Ziel dieser Arbeit ist experimentell nachzuprüfen, ob es ein Beugungsexperiment an einer Dünnschichtsystemprobe mit der gegebenen Röntgenquelle realisierbar ist, und so ein Mess- und Auswertungsverfahren zu entwickeln, in dem einzelne gestreute Photonen mit dem kommerziellen MÖNCH-Detektor \cite{ramilli-measurements-2017}, der durch die hohe Bildrate jeden einzelnen Röntgenpuls aufnehmen lässt.

\noindent
%Damit die geleistete Arbeit in der Zukunft auch als ein Lernmaterial dienen könnte und das Auswertungsverfahren reproduzierbar und nachvollziehbar wäre, werden.
Alle benutzten \href{https://github.com/lrlunin/bachelorarbeit_python_notebooks_and_data}{\color{blue}\texttt{.ipynb} Auswertungsscripte} sowie der \href{https://github.com/lrlunin/bachelorarbeit_lunin_tu_berlin}{\color{blue}\LaTeX-Quellcode} dieses Dokuments werden in die entsprechenden GitHub-Repositories\footnote{\url{https://github.com/lrlunin/bachelorarbeit_python_notebooks_and_data}\newline\url{https://github.com/lrlunin/bachelorarbeit_lunin_tu_berlin}} hochgeladen.   
%Eine besondere Herausforderung
% MOTIVATION SINGLE PHOTON COUNTING

% Idee: für die höhere x-ray Energien ist es möglich die Photonen von dem Rauschen mit einem Komporator (x > threshold) zu trennen. In dem Weichröntgenbereich ist es nicht möglich. Die Innovation liegt daran:  einzelne Photonen in den Aufnahmen mit ultrakurzen Belichtungszeiten zu erkennen, was in Verbindung mit einer hervorragend kurzen Auslesezeit des Detektors ermöglicht, die rauschenlosen Aufnahmen zu bekommen.

% Ich habe noch ganz vergessen, was zu deinem Inhaltsverzeichnis des Theorieteils deiner Arbeit zu sagen. Im Prinzip hast du die richtigen Themen erkannt. Ich würde etwas weiter einengen:
% 1. Resonante Röntgenstreuung von mesoskopischen magnetischen Texturen – bitte lass uns rechtzeitig darüber reden, was da genau rein soll.
% 2. Laser-getriebenes Instrument für resonante Streuung mit weichen Röntgenstrahlen – hier benötigst du fast ausschließlich das Paper von Daniel (https://www.osapublishing.org/optica/abstract.cfm?uri=optica-8-9-1237). Einfach die technischen Details zusammenfassen.
% 3. Der MÖNCH-Detektor – Hier bitte die Paper von den PSI-Leuten nutzen. Nicht zu viel allgemeines, lieber genau den MÖNCH-Detektor erklären.
% 4. Droplet Algorithmen – kurze Übersicht und dann genauer ein Algorithmus, den du ausprobieren möchtest.
% e Qualität bietet sich das Signal-Rausch-Verhältnis (engl. signal-to-noise-ratio, kurz: SNR) 


